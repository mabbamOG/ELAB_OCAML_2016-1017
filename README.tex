% Created 2017-06-11 Sun 16:00
% Intended LaTeX compiler: pdflatex
\documentclass[11pt]{article}
\usepackage[utf8]{inputenc}
\usepackage[T1]{fontenc}
\usepackage{graphicx}
\usepackage{grffile}
\usepackage{longtable}
\usepackage{wrapfig}
\usepackage{rotating}
\usepackage[normalem]{ulem}
\usepackage{amsmath}
\usepackage{textcomp}
\usepackage{amssymb}
\usepackage{capt-of}
\usepackage{hyperref}
\usepackage[margin=0.5in]{geometry}
\usepackage{tabularx}
\author{Mario A. Barbara \\ Andrea Cancellieri}
\date{}
\title{giaco.ml\\\medskip
\large an interpreted EDSL written in OCAML}
\hypersetup{
 pdfauthor={Mario A. Barbara \\ Andrea Cancellieri},
 pdftitle={giaco.ml},
 pdfkeywords={},
 pdfsubject={},
 pdfcreator={Emacs 25.2.1 (Org mode 9.0.7)}, 
 pdflang={English}}
\begin{document}

\maketitle
\section{Introduction}
\label{sec:org965e98b}
giaco.ml is an imperative (but also functional) EDSL interpreter written in OCAML.

The interpreter is capable of handling expressions, commands and declarations all within a program.

A program can be parsed into valid code from a file/string thanks to reflection.

Static taint analysis can be performed, guards have been also
added to the Reflect command as a form of protection, similar to perl.

The main execution functions are: \texttt{interpret} (or \texttt{interpret'} which returns also the declarations' output), \texttt{reflect} and \texttt{taint\_analysis} which all operate on a \texttt{Prog(ds,cs)} construct.
\section{Design Choices}
\label{sec:orgc0160ce}
The interpreter works with 3 domains:
\begin{description}
\item[{generic}] this domain is shared amongst the syntax (external) and semantic (internal) domains
\item[{syntax}] this domain contains all the domains accessible to the user. it features 3 sub-domains
\begin{description}
\item[{expressions}] these are entities which may be mapped directly to a value
\item[{commands}] these are entities which allow modifying the global state of the program, though allocating new memory is not allowed
\item[{declarations}] these are entities which allow allocation of new memory in the global state
\end{description}
\item[{semantic}] this domain is used for representing the internal state of the interpreter.
\begin{description}
\item[{evaluatable values}] these are values that may be directly evaluated from an expressions. they represent the base internal type for the interpreter. Basic allowed types are Integer, Boolean, Float, String and Subprograms.
\item[{environment}] this is a mapping representing internal non-mutable state. It maps variables (identifiers) to values.
\item[{store}] this is a mapping representing internal mutable state. It allows aliasing of 2 identifiers for one value. It maps addresses (pointers or locations) to values.
\item[{denotable values}] these are values contained in the environment. Basic allowed types are  Integer, Boolean, Float, String and Locations.
\item[{mutable memory values}] these are values contained in the store. Basic allowed types are Integer, Boolean, Float, String.
\end{description}
\end{description}

The interpreter itself is implemented by 3 main functions:
\begin{description}
\item[{eval}] this function basically maps expressions to evaluatable values
\item[{cval}] this function maps commands to a changed internal state (store)
\item[{dval}] this function maps declarations to a new internal state (environment + store)
\end{description}

Some important choices:
\begin{itemize}
\item memory is viewed as a mapping. it is therefore impossible to modify in ocaml. changing the type signature of the interpreter would allow this
\item because memory is a mapping, new locations are simulated by randomly selecting a new location based on given store size
\item regular expressions are used in reflection, so the Str module must be available
\item fixed points are calculated through ocamls's \texttt{function} and \texttt{let rec} operations
\item to ease debugging, the interpreted is not compiled but interpreted in an ocaml shell with the \texttt{\#use "file.ml"} construct
\item since the program is interpreted, \texttt{let and} between multiple files is not available, this has been fixed by using mutable state pointers to future functions. These pointers will be updated
when the function is created.
\item many utility functions have been created to ease debugging. most notable are \texttt{emptyenv} \texttt{emptystore} \texttt{env'} \texttt{store'} \texttt{new'} which simplify dealing with memory
\item due to the non-recursive nature of \texttt{Procedure} and \texttt{Call}, they have been omitted from reflection and taint-analysis. The syntax restriction is due to the difficulty of implementing a parser 
for the complete OCAML list syntax, and the static analysis of commands that cannot be recursively unpacked (one lies in the \emph{expressions}, the other in the \emph{commands})
\end{itemize}
\section{Usage}
\label{sec:orgdbdcb5f}
the interpreter can be loaded up with
\begin{verbatim}
>> #use "giaco.ml";;
\end{verbatim}

and subsequent testing may be performed:
\begin{verbatim}
>> (* have a functional sample: *)
>> let e = Plus(Int(3), Int(4)) in eval e emptyenv emptystore;;
- : evalue = EInt 7

>> (* now let's try a complete program *)
>> let e = Lambda("x",Multiply(Val("global"), Var("x"))) in
>> let d = New("global", Str("yes")) in
>> let c = Assign("global", Apply(e, Int(5))) in
>> let program = Prog(d,c) in
>> let result_environment, result_store = interpret' program emptyenv emptystore in
>> eval (Val("global")) result_environment result_store;;
- : evalue = EStr "yesyesyesyesyes"
\end{verbatim}

The most useful interpreter functions are:
\begin{description}
\item[{\texttt{eval: expr -> env -> store -> evalue}}] evaluates expressions
\item[{\texttt{cval: com -> env -> store -> store}}] converts commands into a modified mutable memory
\item[{\texttt{dval: dec -> env -> store -> env*store}}] allows extending mutable memory
\item[{\texttt{interpret: prog -> env -> store -> store} }] combines all of the above. cannot return expression values like eval, though, as we do not have print
functionality
\item[{\texttt{interpret': prog -> env -> store -> env*store} }] just like \texttt{interpret} but return the last evaluated environment as well, which allows for further 
analysis of the output
\item[{\texttt{emptyenv} and \texttt{emptystore}}] already initialized empty environment and store
\item[{\texttt{env'} and \texttt{store'}}] allow extending environments and stores outside of syntax. \texttt{new'} combines this and allows doing something like the \texttt{New(...)} command outside of syntax.
\item[{\texttt{ereflect} and \texttt{creflect} and \texttt{dreflect} and \texttt{reflect}}] reflection of expressions, commands, declarations, full blown programs.
\item[{\texttt{etaint} and \texttt{ctaint} and \texttt{dtaint} and \texttt{taint\_anlaysis}}] static taint analysis of expressions, commands, declarations and full blown programs.
\end{description}

\subsection{EXPRESSIONS}
\label{sec:org85b77ba}
\begin{center}
\begin{tabularx}{\textwidth}{lX}
EXAMPLE & DESCRIPTION\\
\hline
Int(3) & basic integer\\
Str("hello world") & basic ASCII string\\
Bool(true) & basic boolean\\
Float(4.5) & basic float\\
Lambda("x", <exp containing x>) & typical function\\
RecLambda("f", "x", <exp containing f and x>) & typical recursive function\\
Rec("f", Lambda(\ldots{}.)) & just another way to define recursive lambdas\\
Proc(["x";"y";"z";\ldots{}], Block(\ldots{})) & this is a procedure, check the commands section\\
IfThenElse(Bool(true), .., ..) & control flow element\\
Var("x") & this is a way to retrieve an immutable variable's content\\
LetIn("x", e1, e2) & this is a way to nest functional blocks and scopes\\
Val("x") & this is a way to retrieve a mutable variable's content\\
Plus(e1, e2) & plus function, applies to: Int, Str, Float\\
Multiply(e1, e2) & multiply function, appliest to: Int, Str, Float\\
Apply(e1, e2) & typical function application, e1 is of type: Lambda, RecLambda, Rec\\
Equals(e1, e2) & like C's \texttt{==}\\
Greater(e1, e2) & like C's \texttt{>}\\
Not(e) & like C's \texttt{!}\\
Or(e1, e2) & like C's \texttt{¦¦}\\
And(e1, e2) & like C's \texttt{\&\&}\\
Len( Str(\ldots{})) & gets the length of a St\\
Sub(Str(\ldots{}), i, j) & gets a substring. i and j of type Int.\\
Lower(Str(..)) & reduces a string to lowercase, like Python's \texttt{lower()}\\
Upper(Str(\ldots{})) & reduces a string to uppercase, like Python's \texttt{upper()}\\
Trim(Str(\ldots{})) & trims whitespace from a string, like Python's \texttt{s.trim()}\\
Replace(<string to be replace>,<replacer string>,<string>) & replaces a string with another string in a string, like Python's \texttt{s.replace()}\\
\end{tabularx}
\end{center}
\subsection{COMMANDS}
\label{sec:org9c3e8af}
\begin{center}
\begin{tabularx}{\textwidth}{lX}
EXAMPLE & DESCRIPTION\\
\hline
Assign("x", e) & this changes the mutable value for the variable "x". e is an expression\\
Block(d, c) & this is an imperative block with nested scope. d is a declaration, see its section for more detail\\
Call(p, [e1;e2;e3;..]) & this is an application of an imperative procedure. p is of type Proc (check the expressions section)\\
While(e, c) & like C's \texttt{while(e)\{c\}}, e is an expression and c a command\\
CIfThen(e, c) & like C's \texttt{if(e)\{c\}}\\
CIfThenElse(e, c1, c2) & like C's \texttt{if(e)\{c1\}else\{c2\}}\\
CSeq(c1, c2) & like C's \texttt{;} it allows concatenation of commands\\
CSkip & like C's \texttt{void} and Python's \texttt{pass}, it does nothing\\
Reflect(Str(\ldots{})) & reflection, see the reflection section\\
\end{tabularx}
\end{center}
\subsection{DECLARATIONS}
\label{sec:org677b58f}
\begin{center}
\begin{tabularx}{\textwidth}{lX}
EXAMPLE & DESCRIPTION\\
\hline
New("x", e) & this allocates a new mutable variable of value e (an expression)\\
DSeq(d1, d2) & allows concatenation of declarations\\
DSkip & does nothing\\
\end{tabularx}
\end{center}
\subsection{PROGRAM}
\label{sec:orgf8fb6f2}
NOTE :: to use the functional language only, please use expressions along with \texttt{eval}. Programs are only imperative.
\begin{center}
\begin{tabularx}{\textwidth}{ll}
EXAMPLE & DESCRIPTION\\
\hline
Prog(<declaration>, <command>) & handy syntax that puts it all together\\
\end{tabularx}
\end{center}
\section{String extension}
\label{sec:org45020ea}
A few functions have been added to deal with the domain of strings.
Functions such as these are taken from the Python langauge, which has a \textbf{very} extensive and popularstandard library.
\begin{itemize}
\item length comparison (\texttt{Greater})
\item concatenation (\texttt{Plus} has been extended to allow this)
\item substring (\texttt{Sub})
\item repetition (\texttt{Multiply} has been extended to allow this)
\item length (\texttt{Len})
\item lowercase (\texttt{Lower})
\item uppercase (\texttt{Upper})
\item trim (\texttt{Trim}), trims all whitespace
\item Replace (\texttt{Replace})
\end{itemize}

Check the examples section for some examples
\section{Reflection extension}
\label{sec:org82a2a16}
Reflection consists of allowing any string to be evaluated by the interpreter on the fly. In Python this is akeen to the \texttt{eval} function.
This is also the most essential step to having a good interpreter: the interactive console for Python,
 one of the most popular interpreted languages, is often called \textbf{R.E.P.L.} (Read Eval Print Loop).

The syntax of \emph{giaco.ml} has been extended with the \texttt{Reflect} command, which allows on the fly "evaluation" of commands.
Unfortuately our language's command syntax is recursive, and furthermore the \texttt{CIfThenElse} command
uses expressions as boolean conditions, which are also recursive.
Therefore, a full blown parser needed to be built to give a string some depth (such as that of an AST).

The reasoning is as follows:
\begin{enumerate}
\item a function called \texttt{next\_unit} is charged with grabbing the first word up until a \texttt{(} or \texttt{)} or \texttt{,} or multiple consecutive repetitions.
\item to get the command to match against, \texttt{next\_unit} is called on the string and the result is matched against some constants, taken from the language's syntax
\item to get a command's arguments (which may be recursive and contain any amount of \texttt{( ) ,}, caution must be taken to correctly identify the argument boundaries, which are all separated by a \texttt{,} comma. 2 options are given:
\begin{description}
\item[{iterative}] by counting the number of open parentheses matched thus far, and decreasing each time a closed parentheses is found,
it is possible to correctly identify the recursive structure of the syntax.
\item[{recursive (but faster)}] since we know the amount of parameter each command needs, it is
simply required to recursively reflect upon the arguments' string as many times as needed.
Of course, each time a command is consumed, it shall return the arguments' string, so as to allow its father to continue
looking for arguments.
\end{description}
\end{enumerate}

Our interpreter implements the recursive and faster technique. Here is a simple ditaa drawing to illustrate
the flow of this technique:

\begin{verbatim}
+------------------------------+
|  A(B(C(1), C(2), C(3), ...)) |
|                              |
+---------------+--------------+
                |
                v

+---+---------------------------+
| A | B   C   1   C   2
+-------+-----------------------+
    | B | C   1   C   2           <----+
    +-------+-------------------+      |
        | C | 1   C   2                |
        +-------+---------------+      |
            | 1 | C   2       |--------+
        --------+---------------+      |
        | C | 2                        |
        +-------+---------------+      |
            | 2 |             |--------+
            +---+---------------+
\end{verbatim}
\section{Taint-Analysis extension}
\label{sec:org900e0d5}
Static taint analysis consists of understanding how much damage some unsafe elements (of undefined value but defined nature) will yield.
A classic example is an unsanitized input on a HTML form, which may result in an SQL Injection attack and damage your company's most valuable assets.

In our simple language, we have no operations that deal with the outside world (yet). We are thereforce forced to ask the user to label some variables
in the environment and store as \texttt{Clean} or \texttt{Dirty}. Afterwards, we will analyze a program and check the \texttt{Taint} for every possible variable assignment.
The semantic domains have been revisited, allowing memory (environment and store) to only contain tainted values (or store locations, in the environment's case).

Our analysis is based on 2 simple concepts:
\begin{description}
\item[{pure evaluation}] a \texttt{tor} function will take 2 taints and return \texttt{Dirty} if one of them is as well, otherwise \texttt{Clean}. This process can be lazy.
\begin{itemize}
\item All constants are \texttt{Clean}
\item If a function is involved (such as a \texttt{Lambda}) then the formal parameters are identified as \texttt{Clean} (as they cannot be expressions) and then the body is analyzed. If the body is clean, the function is clean
\item A function application requires a \texttt{tor} amongst the analysis of the function itself and the passed parameter.
\item If a condition is involved, then 2 outputs are possible. If the condition is \texttt{Dirty}, that means the attacker may choose either output and (regardless of the output's default taint)
will result in a \texttt{Dirty} value. If the condition is \texttt{Clean}, then either output may occur during execution, so they must be passed to \texttt{tor}.
\end{itemize}
\item[{imperative state change}] all possible assignments in a command are gathered. Only the latest possible assignments matter (if i set \texttt{x} to \texttt{Dirty} and then \texttt{Clean} it is \texttt{Clean}).
\begin{itemize}
\item Afterwards, we check whether 2 branches are possible: if they are, a \texttt{tor} function must be applied to all assignments of same key, merging the 2 branches.
\item If the branches are subject to a condition (such as in a \texttt{CIfThenElse}) then a \texttt{Dirty} condition will mean an attacker may choose amongst any of the 2 branches, therefore dirtying all 
assignments of shared key (amongst the 2 branches). If the condition is \texttt{Clean}, then the normal merge has already evaluated taint with \texttt{tor}.
\end{itemize}
\end{description}
\section{Examples}
\label{sec:orgbf15a20}
check test.ml for some code examples.
\subsection{Numbers}
\label{sec:orgb785289}
\begin{center}
\begin{tabularx}{\textwidth}{XX}
INPUT & OUTPUT\\
\hline
Int(5) & EInt 5\\
Float(133.7) & EFloat 133.7\\
Plus(Int(1), Int(2)) & EInt 3\\
Multiply(Float(2.5),Float(10.0)) & EFloat 25\\
Greater(Int(3),Int(5)) & EBool false\\
\end{tabularx}
\end{center}
\subsection{Booleans}
\label{sec:org87187cd}
\begin{center}
\begin{tabularx}{\textwidth}{Xl}
INPUT & OUTPUT\\
\hline
Bool(true) & EBool true\\
Not(Bool(true)) & EBool false\\
And(Equals(Float(4.5),Float(4.6)),Equals(Float(0.1),Float(0.1))) & EBool false\\
Or(Equals(Float(4.5),Float(4.6)),Equals(Float(0.1),Float(0.1))) & EBool true\\
\end{tabularx}
\end{center}
\subsection{Strings}
\label{sec:orgd0e160b}
\begin{center}
\begin{tabularx}{\textwidth}{XX}
INPUT & OUTPUT\\
\hline
Str("hello world") & EStr "hello world"\\
Plus(Str("hello "),Str("world!")) & EStr "hello world!"\\
Multiply(Str("abc"),Int(10)) & EStr "abcabcabcabcabcabcabcabcabcabc"\\
Len(Multiply(Str("abc"),Int(10))) & EInt 30\\
Greater(Str("two"),Str("three")) & EBool false\\
Sub(Str("threeeeeeee"),Int(2),Int(10)) & EStr "reeeeeeee"\\
Upper(Str("im so lonely")) & EStr "IM SO LONELY"\\
Lower(Upper(Str("im so lonely"))) & EStr "im so lonely"\\
Trim(Str("     italia      ")) & EStr "italia"\\
Replace(Str("hello"),Str("goodbye"),Str("hello world!")) & EStr "goodbye world!"\\
\end{tabularx}
\end{center}
\subsection{Functional Control Flow}
\label{sec:org6ac5d62}
\begin{center}
\begin{tabularx}{\textwidth}{Xl}
INPUT & OUTPUT\\
\hline
IfThenElse(Bool(true), Int(1337), Str("i am")) & EInt 1337\\
IfThenElse(Not(Greater(Str("bob"),Str("mouse"))),Str("ciao mondo"),Int(5)) & EStr "ciao mondo"\\
\end{tabularx}
\end{center}
\subsection{Functional Blocks}
\label{sec:org35c5a0f}
\begin{center}
\begin{tabularx}{\textwidth}{Xl}
INPUT & OUTPUT\\
\hline
Var("x") & EInt 20\\
\texttt{xxx =} LetIn("a",Int(3),Multiply(Var("a"),Var("a"))) & EInt 9\\
LetIn("a",Int(5),(LetIn("b",xxx,LetIn("c",Int(6),Plus(Var("a"),Plus(Var("b"),Var("c"))))))) & EInt 20\\
\end{tabularx}
\end{center}
\subsection{Functional Subprograms}
\label{sec:org08c40aa}
\begin{center}
\begin{tabularx}{\textwidth}{Xl}
INPUT & OUTPUT\\
\hline
Apply(Lambda("x", Plus(Var("x"), Int(1))), Int(99)) & EInt 100\\
Apply(RecLambda("fact", "x", IfThenElse(Equals(Var("x"), Int(0)), Int(1), Multiply(Var("x"), Apply(Var("fact"), Plus(Var("x"), Int(-1)))))), Int(10)) & EInt 3628800\\
\end{tabularx}
\end{center}
\subsection{Imperative State Change}
\label{sec:org962f0cb}
\begin{center}
\begin{tabularx}{\textwidth}{ll}
INPUT & VARIABLE OUTPUT\\
\hline
Val("y") & EInt 10\\
Assign("y", Plus(Val("y"), Val("y"))) & EInt 20\\
\end{tabularx}
\end{center}
\subsection{Imperative Control Flow}
\label{sec:org14863b3}
\begin{center}
\begin{tabularx}{\textwidth}{Xl}
INPUT & VARIABLE OUTPUT\\
\hline
Val("y"), Val("z") & EInt 10, EInt 0\\
CIfThenElse(Not(Equals(Val("y"),Int(11))), Assign("y", Int(50))) & EInt 50, EInt 0\\
While(Not(Equals(Val("y"), Int(100))), CSeq(Assign("y", Plus(Val("y"), Int(1))), Assign("z", Plus(Val("z"), Int(1)))) ) & EInt 50, EInt 50\\
\end{tabularx}
\end{center}
\subsection{Imperative Blocks}
\label{sec:orgf965891}
\begin{center}
\begin{tabularx}{\textwidth}{Xl}
INPUT & OUTPUT\\
\hline
Val("y"), Val("z") & EInt 10, EInt 0\\
Block(New("z", Int(1000)), Assign("y", Plus(Val "y", Val "z"))) & EInt 1010, EInt 0\\
\end{tabularx}
\end{center}
\subsection{Imperative Subprograms}
\label{sec:org4a5830d}
\begin{center}
\begin{tabularx}{\textwidth}{Xl}
INPUT & OUTPUT\\
\hline
Val("y"), Val("z") & EInt 10, EInt 0\\
\texttt{f =} Proc(["z"], Block(DSkip, Assign("y", Val("z"))))) in Call (Val "f", [Val "z"]) & EInt 0, EInt 0\\
\end{tabularx}
\end{center}
\subsection{Declarations}
\label{sec:org5aa12a1}
\begin{center}
\begin{tabularx}{\textwidth}{lX}
INPUT & OUTPUT\\
\hline
Val("y"), Val("z") & Failure 'y' not in environment, Failure 'z' not in environment\\
DSeq(New("y", Int(10)), New("z", Int(0))) & EInt 10, EInt 0\\
\end{tabularx}
\end{center}
\subsection{Reflection}
\label{sec:org078555a}
\begin{center}
\begin{tabularx}{\textwidth}{Xl}
INPUT & OUTPUT\\
\hline
Val("y") & EInt 10\\
\texttt{ereflect} ("Plus(Plus(Int(1), Int(2)), Plus(Int(3), Int(4)))") & EInt 10\\
Reflect(Str("Assign($\backslash$"y$\backslash$", Int(5))")) & EInt 5\\
\end{tabularx}
\end{center}
\subsection{Taint Analysis}
\label{sec:org775053a}
"dirty" is Dirty, "clean" is Clean
\begin{center}
\begin{tabularx}{\textwidth}{lX}
INPUT & VALUE\\
\hline
\texttt{"dirty"} & Dirty\\
\texttt{"clean"} & Clean\\
\texttt{e} & Equals(Plus(Val("x"),Val("y")),Int(6))\\
\texttt{assign1} & CSeq(Assign("x", Val("dirty")), Assign("y", Val("clean")))\\
\texttt{assign2} & CSeq(Assign("x", Val("clean")), Assign("y", Val("dirty")))\\
\texttt{d} & DSeq(New("x", Val("dirty")), New("y", Val("clean")))\\
\texttt{c} & CIfThenElse(e, assign1, assign2)\\
\end{tabularx}
\end{center}

\begin{center}
\begin{tabularx}{\textwidth}{lX}
INPUT & OUTPUT\\
\hline
\texttt{taint\_analysis} Prog(d,c) & [("clean", TLoc 15n); ("dirty", TLoc 27n); ("x", TLoc 76n); ("y", TLoc 41n)]\\
 & [(15n, Clean); (27n, Dirty); (41n, Dirty); (76n, Dirty)]\\
\end{tabularx}
\end{center}
\section{Noted Bugs:}
\label{sec:orgb88727e}
\begin{itemize}
\item reflection on the \texttt{Reflect} command is supposedly allowed (since it is implemented recursively) but OCAML's escape of strings within strings is weird and needs some fixing\ldots{}
\end{itemize}
\end{document}
